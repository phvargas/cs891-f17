%% vi5-template.tex
%% 2017/09/25 - based on
%%
%% bare_jrnl_compsoc.tex
%% V1.4b
%% 2015/08/26
%% by Michael Shell
%% See:
%% http://www.michaelshell.org/
%% for current contact information.
%%
%% This is a skeleton file demonstrating the use of IEEEtran.cls
%% (requires IEEEtran.cls version 1.8b or later) with an IEEE
%% Computer Society journal paper.
%%
%% Support sites:
%% http://www.michaelshell.org/tex/ieeetran/
%% http://www.ctan.org/pkg/ieeetran
%% and
%% http://www.ieee.org/

%%*************************************************************************
%% Legal Notice:
%% This code is offered as-is without any warranty either expressed or
%% implied; without even the implied warranty of MERCHANTABILITY or
%% FITNESS FOR A PARTICULAR PURPOSE!
%% User assumes all risk.
%% In no event shall the IEEE or any contributor to this code be liable for
%% any damages or losses, including, but not limited to, incidental,
%% consequential, or any other damages, resulting from the use or misuse
%% of any information contained here.
%%
%% All comments are the opinions of their respective authors and are not
%% necessarily endorsed by the IEEE.
%%
%% This work is distributed under the LaTeX Project Public License (LPPL)
%% ( http://www.latex-project.org/ ) version 1.3, and may be freely used,
%% distributed and modified. A copy of the LPPL, version 1.3, is included
%% in the base LaTeX documentation of all distributions of LaTeX released
%% 2003/12/01 or later.
%% Retain all contribution notices and credits.
%% ** Modified files should be clearly indicated as such, including  **
%% ** renaming them and changing author support contact information. **
%%*************************************************************************

% *** Authors should verify (and, if needed, correct) their LaTeX system  ***
% *** with the testflow diagnostic prior to trusting their LaTeX platform ***
% *** with production work. The IEEE's font choices and paper sizes can   ***
% *** trigger bugs that do not appear when using other class files.       ***                          ***
% The testflow support page is at:
% http://www.michaelshell.org/tex/testflow/

\documentclass[11pt,journal,compsoc,onecolumn]{IEEEtran}
%
% If IEEEtran.cls has not been installed into the LaTeX system files,
% manually specify the path to it like:
% \documentclass[10pt,journal,compsoc]{../sty/IEEEtran}

\usepackage{cite}
\usepackage[pdftex]{graphicx}
\usepackage{array}
\usepackage{url}

\begin{document}
%
% paper title
% Titles are generally capitalized except for words such as a, an, and, as,
% at, but, by, for, in, nor, of, on, or, the, to and up, which are usually
% not capitalized unless they are the first or last word of the title.
% Linebreaks \\ can be used within to get better formatting as desired.
% Do not put math or special symbols in the title.
\title{Web Collection Report\\
Assignment - Week 6, CS 891, Fall 2017}

\author{Plinio Vargas\\
Department of Computer Science\\
Old Dominion University\\
Norfolk, VA 23529\\
pvargas@cs.odu.edu
}

% The paper headers
\markboth{Vargas, CS 891, Fall 2017}{}

% make the title area
\maketitle

\section{Introduction}
The preservation of events has had a great deal of importance in every culture of human civilization. Prior to written form of communication, historic preservation was narrated from generation to generation using songs or stories to account for great battles that provided identity and a sense of unity to a particular social nucleus. With the appearance of written communication, the preservation of historical accounts became more accurate, providing future generations the ability to go into the past, thus making learning and connecting with their ancestors in a way that was not possible using oral communication. Today, the web presents to our current generation a vast amount of information that has never been seen before. It contains social aspects, political accounts, medical advances, etc. However, the information is so dynamic that its contents  also disappears at a great speed. Archiving the web is a quest to collect current information appearing on the web for preservation. Our group learned of four tools useful for web archiving. This report is an evaluation to those tools, and my experience using them.


\section{The Collection: Las Vegas massacre 2017}At the time this assignment was given, the United States went through the deadliest mass shooting in its history.  On October 1, 2017 a shooter opened fire onto a country music concert crowd from the 32nd floor of the Mandalay Bay hotel in Las Vegas. 58 people lost their lives in this incident. The 15 seeds for the collection are:\\

\begin{itemize}
	\item \scriptsize\url{https://newrepublic.com/article/145192/alt-lite-celebrated-las-vegas-massacre}
	\item \url{https://www.rawstory.com/2017/10/trump-campaign-exploits-las-vegas-massacre-in-fundraising-email/}
	\item \url{http://www.mercurynews.com/2017/10/07/gun-control-debate-ask-las-vegas-massacre-survivors-where-they-stand/}
	\item \url{http://www.latimes.com/local/lanow/la-me-vegas-memorial-20171007-story.html}
	\item \url{https://www.reuters.com/article/us-lasvegas-shooting/pence-offers-solace-as-las-vegas-police-puzzle-over-shooters-motive-idUSKBN1CA0X6}
	\item \url{http://nypost.com/2017/10/07/hero-guards-in-las-vegas-massacre-are-back-to-work/}
	\item \url{http://abcnews.go.com/US/las-vegas-shooting-mother-father-dead/story?id=50229707}
	\item \url{https://www.yahoo.com/news/isis-las-vegas-shooter-stephen-004312350.html}
	\item \url{http://www.aljazeera.com/indepth/opinion/las-vegas-massacre-reveals-troublesome-global-truth-171006054512920.html}
	\item \url{http://www.dailynews.com/2017/10/07/family-of-canyon-country-man-slain-in-las-vegas-massacre-seeks-to-freeze-killers-assets/}
	\item \url{https://www.washingtonpost.com/news/morning-mix/wp/2017/10/02/police-shut-down-part-of-las-vegas-strip-due-to-shooting/?utm_term=.1c25ebc3892e}
	\item \url{https://www.cbsnews.com/news/victims-of-las-vegas-shooting-list-names-latest-update/}
	\item \url{http://abcnews.go.com/US/motive-remains-elusive-days-las-vegas-massacre/story?id=50320760}
	\item \url{http://www.cnn.com/2017/10/05/us/las-vegas-shooting-timeline/index.html}
	\item \url{http://www.cnn.com/2017/10/05/us/inside-the-las-vegas-massacre/index.html}

\end{itemize}

\newpage
\section{Tools}
\subsection{Archive-it.org}
\subsubsection{Advantages}
\begin{itemize}
	\item Archive-it.org has a ``seed'' feature where the user can drop a list of URLs to archive the collection. Here the user may determine the frequency on which the seed will be archived (One-time, daily, weekly, etc.); seed type, which determines if available links in the html page will also be archived (standard is the default); collection access type (public or private); if it provides information about the last time the seed was crawled, the number of captures and  a link of a particular seed to the Wayback machine.
	\begin{figure}[h]
		\includegraphics[scale=.50]{images/seeds.png}
	\end{figure}

	\item The crawl feature tab is very useful. It provides information  on the crawl status to include the starting and completion time, the frequency of the crawls, the collection size (in GB), and the number of documents that have been collected.  I used this feature when I created my collection. I used the default \textbf{standard} seed type which  crawls all the links found in a particular web page and consequent links of the newly discovered pages.  As time passed, the number of documents kept growing, so did the collection size, so I decided to stop the crawl process manually.
	\begin{figure}[h]
		\includegraphics[scale=.50]{images/crawl_report.png}
	\end{figure}
	\newpage
	\item Although, I did not use the \textbf{crawl frequency} feature, exploring the  functionality reveals its importance. The previous point described how fast the collection size can increase if the wrong seed option is selected, so in the crawl frequency tab, the user has the ability to edit the crawl limits, including the number of documents, data and size limits.
	
	\item I made use of the metadata collection functionality. The user has the ability to upload a picture representative of the collection and provide some description related to its content. 
	
	\item Archive-it.org appears to do a good job in preserving a time capsule of side articles relative to the time frame when the story was written, which is the desire effect. A snapshot of the archived page shows that during the time when the massacre occurred, there were also wild fires going on in California and the Vice-President Mike Pence in a PR stunt leaving an NFL game just after the national anthem was played.
	\begin{figure}[h]
		\includegraphics[scale=.45]{images/arcv_abc_good.jpg}
	\end{figure}	
\end{itemize}
\newpage
\subsubsection{Disadvantages}
\begin{itemize}
	\item There were various instances when a very important part of a page was not shown.  In one of the seeds that I selected, my interest was focused on preserving the faces of the people who died in the massacre.  As is shown on the capture below, the archived page failed to save the picture of one of the victims of the massacre.
	\begin{figure}[h]
		\includegraphics[scale=.50]{images/arcv_abc.png}
	\end{figure}
\end{itemize}
	
\subsection{Webrecorder.io}
\subsubsection{Advantages}
\begin{itemize}
	\item If the user is not interested in capturing the entire page, but just a particular portion of it, then this is the best tool for that type of requirement.
	
	\item A feature not available in \textbf{Archive-it.org}, but available in \textbf{Webrecorder} is the ability to add WARC files to the collection.
	
	\item The user also has the option of downloading the entire collection to the local computer using WARC formatted files.

	\item When viewing a collection, it can be organized in such a form that it may be replayed in the form of a story. See snapshot below.
	
	\begin{figure}[h]
		\includegraphics[scale=.50]{images/replay_option.png}
	\end{figure}
	
	\newpage
	\item \textbf{Webrecorder.io} has a replay option which helps to patch a file that doesn't render properly. This is an option that \textbf{Archive-it.org} doesn't have and it proves to be very useful. Using my collection as an example, the archived file containing the pictures of mortally wounded individuals in the massacre is missing the photo of one individual. The ability to patch and fix a very important portion of a file helps the user to preserve the intended content in any particular collection.
\end{itemize}

\subsubsection{Disadvantages}
\begin{itemize}
	\item \textbf{Webrecorder.io}'s recording interaction feature  with a user may also become a major disadvantage, specially in the case when the user is required to download links of any particular web page. Every time there is user interaction, there is a possibility for error. Therefore, as the number of links in a page increases, so does the possibility that the user may not click on a very important link, and not including it into the collection.
	
	\item The interface bar on \textbf{Webrecorder.io} keeps flashing recording, but there there is not a way to know if all the resources representation have been loaded. The user has to go and replay the saved resource and verify that all tags render properly.
	
\end{itemize}

\subsection{WAIL}
\subsubsection{Advantages}
\begin{itemize}
	\item \textbf{WARC} files can be dropped into the web collection interface
	
	\item The user has the option of archiving an URL directly from the live web.
	
\end{itemize}

\subsubsection{Disadvantages}
\begin{itemize}
	\item When dragging WARC files into the WAIL interface the user only can drag and drop one file at a time.  If additional WARC files were to be added the previous file waiting in the queue to be archived will disappear. A better option will be drag all the desired files into the interface and click on the available select button to add all the WARC seeds files.
	
	\begin{figure}[h]
		\includegraphics[scale=.25]{images/wail.png}
	\end{figure}
	
	
\end{itemize}

\subsection{WARCreate}
\subsubsection{Advantages}
\begin{itemize}
	\item Representation of a web page resources can be saved into a compressed format. The user does not have to worry about the resource size. If required, \textbf{WARCreate} will break the file into various chunks to accommodate size.
	
	\item All resources representation are retrieved automatically without any user intervention.
	
	\item The steps to create a WARC file is very simplify to the user. The user opens Chrome web browser, types the URL and clicks on \textbf{�}{WARCreate} add-on
	
\end{itemize}

\newpage
\subsubsection{Disadvantages}
\begin{itemize}
	\item It is only available in Chrome
	
	\item A major disadvantage of \textbf{WARCreate} is the lack of interaction with the user during the generation of the WARC file. If the WARC file is not generated quickly when the user presses the ``Generate WARC'' button, then the user may continue to press the same button several times, leading to the duplication of the same WARC file. This was a common problem while working with \textbf{WARCreate}.
	\begin{figure}[h]
		\includegraphics[scale=.25]{images/warcreate_prob.jpg}
	\end{figure}	

\newpage	
	\item There is no way to patch a particular resource representation into the WARC file. The two screen-shots below are taken from \textbf{Webrecorder.io} and \textbf{WAIL} respectively. WAIL archived a WARC file generated by \textbf{WARCreate}. It can be appreciated that the \textbf{Webrecorder.io} representation includes a video related to the massacre while the second representation does not.
		\begin{figure}[h]
		\includegraphics[scale=.25]{images/webrec_wail.png}\\
		\includegraphics[scale=.25]{images/wail_webrecord.png}
	\end{figure}
\end{itemize}

\newpage
\section{Conclusions}
I believe that 15 seeds were enough to preserve what transpired in the Las Vegas massacre. Heroism came to light on videos showing first responders providing medical assistance to injured victims. A web page was added to the collection showing the faces of those who were mortally wounded. The conversation about where people stand on gun control was preserved. The healing process was captured when the Vice-President and the President paid visit to the victims of this horrific act. Fake news angle also was preserved on how  the Islamic State militant group (ISIS) published information detailing how they were responsible for this attack.

A few weeks after the Las Vegas massacre, the archived page compared with the live page shows some difference in content.  The screen-shot below is from the original archived page \url{http://www.mercurynews.com/2017/10/07/gun-control-debate-ask-las-vegas-massacre-survivors-where-they-stand/}. The section labeled ``Related Articles'' marks some of the differences between the two represented resources. 
	\begin{figure}[h]
		\includegraphics[scale=.50]{images/arcv_original.png}
	\end{figure}	
The archived page shown in this section the following articles: American Horror Story - Vegas tragedy forces edits to scene; Many in country music mum over gun issues after Vegas deaths; Where did Las Vegas shooter Stephen Poddock buy his gun?; At Reno gun show, firearms fan denounce gun control but support bump stock ban; A liberal teacher became a conservative enemy for viral Vegas tweet. But does she actually exists?

On the other hand, the live web still shows stories related to the Las Vegas massacre, but they are different: California cops injured in Las Vegas mass shooting heroism denied workers' comp due to state law; father of California woman killed in Las Vegas shooting files wrongful-death lawsuit
; Stanford University to study body of Vegas shooter; Las Vegas officials again adjust shooting timeline,  GOP punts �bump-stock� decision to ATF.

	\begin{figure}[h]
		\includegraphics[scale=.50]{images/arcv_present.png}
	\end{figure}

\end{document}


